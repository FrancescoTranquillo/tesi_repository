\documentclass[10pt,]{article}
\usepackage{lmodern}
\usepackage{setspace}
\setstretch{1.5}
\usepackage{amssymb,amsmath}
\usepackage{ifxetex,ifluatex}
\usepackage{fixltx2e} % provides \textsubscript
\ifnum 0\ifxetex 1\fi\ifluatex 1\fi=0 % if pdftex
  \usepackage[T1]{fontenc}
  \usepackage[utf8]{inputenc}
\else % if luatex or xelatex
  \ifxetex
    \usepackage{mathspec}
  \else
    \usepackage{fontspec}
  \fi
  \defaultfontfeatures{Ligatures=TeX,Scale=MatchLowercase}
\fi
% use upquote if available, for straight quotes in verbatim environments
\IfFileExists{upquote.sty}{\usepackage{upquote}}{}
% use microtype if available
\IfFileExists{microtype.sty}{%
\usepackage{microtype}
\UseMicrotypeSet[protrusion]{basicmath} % disable protrusion for tt fonts
}{}
\usepackage[asymmetric]{geometry}
\usepackage{hyperref}
\hypersetup{unicode=true,
            pdfborder={0 0 0},
            breaklinks=true}
\urlstyle{same}  % don't use monospace font for urls
\usepackage{graphicx,grffile}
\makeatletter
\def\maxwidth{\ifdim\Gin@nat@width>\linewidth\linewidth\else\Gin@nat@width\fi}
\def\maxheight{\ifdim\Gin@nat@height>\textheight\textheight\else\Gin@nat@height\fi}
\makeatother
% Scale images if necessary, so that they will not overflow the page
% margins by default, and it is still possible to overwrite the defaults
% using explicit options in \includegraphics[width, height, ...]{}
\setkeys{Gin}{width=\maxwidth,height=\maxheight,keepaspectratio}
\IfFileExists{parskip.sty}{%
\usepackage{parskip}
}{% else
\setlength{\parindent}{0pt}
\setlength{\parskip}{6pt plus 2pt minus 1pt}
}
\setlength{\emergencystretch}{3em}  % prevent overfull lines
\providecommand{\tightlist}{%
  \setlength{\itemsep}{0pt}\setlength{\parskip}{0pt}}
\setcounter{secnumdepth}{5}
% Redefines (sub)paragraphs to behave more like sections
\ifx\paragraph\undefined\else
\let\oldparagraph\paragraph
\renewcommand{\paragraph}[1]{\oldparagraph{#1}\mbox{}}
\fi
\ifx\subparagraph\undefined\else
\let\oldsubparagraph\subparagraph
\renewcommand{\subparagraph}[1]{\oldsubparagraph{#1}\mbox{}}
\fi

%%% Use protect on footnotes to avoid problems with footnotes in titles
\let\rmarkdownfootnote\footnote%
\def\footnote{\protect\rmarkdownfootnote}

%%% Change title format to be more compact
\usepackage{titling}

% Create subtitle command for use in maketitle
\providecommand{\subtitle}[1]{
  \posttitle{
    \begin{center}\large#1\end{center}
    }
}

\setlength{\droptitle}{-2em}

  \title{}
    \pretitle{\vspace{\droptitle}}
  \posttitle{}
    \author{}
    \preauthor{}\postauthor{}
    \date{}
    \predate{}\postdate{}
  
\usepackage[italian]{babel}
\usepackage{placeins}
\usepackage{setspace}
\usepackage{chngcntr}
\onehalfspacing
\counterwithin{figure}{section}
\counterwithin{table}{section}
\evensidemargin=0in
\oddsidemargin=0.5in

\begin{document}

\pagenumbering{gobble}

\thispagestyle{empty} \vspace*{-2cm}

\begin{center}
  \large
  POLITECNICO DI MILANO\\
  \large
  Corso di Laurea Magistrale in Ingegneria Biomedica\\
  Dipartimento di Elettronica, Informazione e Bioingegneria\\
  \vspace*{1cm}
  \begin{figure}[htbp]
    \begin{center}
      \includegraphics[width=4cm]{img/polimi}
%   \psfig{file=img/polimi.png,width=3.5cm}
    \end{center}
  \end{figure}
  \vspace*{1cm} \LARGE



  \textbf{LA MANUTENZIONE DELLE APPARECCHIATURE BIOMEDICHE NELL'ERA DELLE TECNOLOGIE DIGITALI}\\




\end{center}

\vspace*{1cm} \large

\begin{flushleft}


  Relatore: Prof. Veronica Cimolin \\
  Correlatore: Ing. Daniela Motta 

\end{flushleft}

\vspace*{1.0cm}

\begin{flushright}


  Tesi di Laurea di:\\ Francesco Raffaele Tranquillo, matricola 905980 \\


\end{flushright}

\vspace*{0.5cm}

\begin{center}



  Anno Accademico 2018-2019
\end{center}

\cleardoublepage

\vspace{17cm}

\large

\begin{flushright}
\itshape{ Spazio per dedica}
\end{flushright}

\cleardoublepage
\pagenumbering{Roman}

\section*{Sommario}\label{sommario}
\addcontentsline{toc}{section}{Sommario}

Il sommario deve contenere 3 o 4 frasi tratte dall'introduzione di cui
la prima inquadra l'area dove si svolge il lavoro (eventualmente la
seconda inquadra la sottoarea più specifica del lavoro), la seconda o la
terza frase dovrebbe iniziare con le parole ``Lo scopo della tesi
è\ldots{}'' e infine la terza o quarta frase riassume brevemente
l'attività svolta, i risultati ottenuti e deventuali valutazioni di
questi.

\pagebreak

\section*{Abstract}\label{abstract}
\addcontentsline{toc}{section}{Abstract}

The abstract must contains 3 or 4 sentences from the introduction. The
first one should be related to the area of the study, with the second
one more, possibly, specific about the same area. The third one should
start with the formula: ``The goal of this dissertation is\ldots{}''.
Finally, the fourth sentence should be a brief summary of the activity,
with the relatives results and possible evaluation of the same.
\pagebreak
\tableofcontents
\newpage
\listoffigures
\newpage
\listoftables

\pagebreak
\pagenumbering{arabic}

\section{Manutenzione predittiva}\label{manutenzione-predittiva}

In questo capitolo verranno prima trattati gli aspetti teorici relativi
alla manutenzione predittiva e successivamente le reali possibilità di
applicazione della stessa nella realtà ospedaliera della ASST di
Vimercate.

Tra gli aspetti teorici trattati, la
\protect\hyperlink{ux5cux23ux5cux2520Origineux5cux2520storicaux5cux2520eux5cux2520inquadramentoux5cux2520normativo}{prima
parte} sarà dedicata all'inquadramento storico e quello normativo
italiano che definisce la manutenzione predittiva. La
\protect\hyperlink{ux5cux23ux5cux2520Metodiux5cux2520analiticiux5cux2520diux5cux2520predizione}{seconda
parte} si concentrerà sulle metodiche di analisi utilizzate in
manutenzione predittiva.

Nella seconda parte del capitolo verrà descritto l'iter perseguito,
durante i mesi di tirocinio, al fine di individuare ipotetiche soluzioni
di manutenzione predittiva da implementare nell'ospedale di Vimercate,
definendo i risultati ottenuti a fronte dei limiti incontrati.

\subsection{Origine storica e inquadramento
normativo}\label{origine-storica-e-inquadramento-normativo}

\subsubsection{Industria 4.0 e tecnologie
emergenti}\label{industria-4.0-e-tecnologie-emergenti}

La storia della manutenzione predittiva è intrinsecamente legata a
quella della quarta rivoluzione industriale la quale, a sua volta, si
configura come uno sviluppo della terza rivoluzione industriale,
definita come rivoluzione digitale. Quest'ultima, iniziata negli anni 80
del secolo scorso, è caratterizzata dalle innovazioni tecnologiche che
hanno permesso il ``salto tecnologico'' dalle tecnologie analogiche e
dei dispositivi meccanici alle attuali tecnologie digitali, come ad
esempio il pc (personal computer), internet e in generale la branca
degli argomenti di interesse della ICT (information and communications
technology). In questo contesto, la quarta rivoluzione industriale nasce
proprio da queste innovazioni tecnologiche ed è caratterizzata
dall'evoluzione di tecnologie emergenti il cui impatto sulla società e
sulla qualità della vita non ha eguali in tutta la storia umana. Tra le
tecnologie emergenti rientrano la robotica, la nanotecnologia, i
computer quantistici, la medicina rigenerativa, l'Industrial Internet of
Things, la domotica e l'intelligenza artificiale applicata in svariati
campi (per esempio automazione industriale, diagnostica per immagini,
buisness intelligence e analisi di big data).

Con questa premessa, si intuisce come la manutenzione predittiva sia
diretta conseguenza di un' applicazione sinergica delle sopracitate
tecnologie. Essa si appropria infatti di metodiche caratteristiche di
diversi campi al fine di determinare lo stato di salute di una
tecnologia per prevedere l'istante temporale ottimale in cui condurre le
operazioni di manutenzione e quindi il tempo residuo prima di un guasto.

Tra queste metodiche rientra ad esempio l'utilizzo di tecnologie IoT:
infatti la valutazione dello stato di salute di un ipotetico parco
macchine viene effettuata tramite l'utilizzo di una rete di sensori in
grado di comunicare l'andamento nel tempo di alcune variabili di
interesse (monitoraggio online). Oppure ancora, come il nome stesso
suggerisce, la componente ``predittiva'' è affidata a più o meno
sofisticati, a seconda del contesto, algoritmi di artificial
intelligence basati a loro volta sull'applicazione di tecniche di
machine learning in grado, in questo caso, di analizzare e
predirel'evoluzione di serie temporali sia in modo semi-automatico
(apprendimento supervisionato) sia in modo totalmente automatico
(apprendimento non supervisionato).

\subsubsection{Il contesto normativo}\label{il-contesto-normativo}

Dal punto di vista normativo, la definizione di manutenzione predittiva
viene delineata, a livello europeo, nella EN 13306 dove, nella versione
attualmente in vigore (EN 13306:2017) essa viene definita come:
\emph{``Condition based maintenance carried out following a forecast
derived from repeated analysis or known characteristics and evaluation
of the significant parameters of the degradation of the item''(Schmidt,
Sandberg, and Wang, n.d.).}

La stessa viene recepita in Italia con la UNI EN 13306:2018, secondo la
quale per ``manutenzione predittiva'' si intende: \emph{``Manutenzione
su condizione eseguita in seguito a una previsione derivata dall'analisi
ripetuta o da caratteristiche note e dalla valutazione dei parametri
significativi afferenti il degrado dell'entità''(Maccarelli, n.d.).}
Dove, sempre secondo la stessa norma, la manutenzione su condizione è
definita come: \emph{``Manutenzione preventiva che comprende la
valutazione delle condizioni fisiche, l'analisi e le possibili azioni di
manutenzione conseguenti''.}

La ``valutazione'', sempre secondo la sopracitata norma, può avvenire
mediante diverse modalità tra le quali:

\begin{itemize}
\tightlist
\item
  Osservazione dell'operatore
\item
  Ispezione
\item
  Collaudo
\item
  Monitoraggio delle condizioni dei parametri del sistema
\end{itemize}

Tutte queste modalità vengono intese come ``svolte secondo un programma,
su richiesta o in continuo''.

Riassumendo, quindi, la manutenzione predittiva si configura come un
caso ``avanzato'' di manutenzione preventiva, che mira alla
minimizzazione dei tempi di fermo macchina grazie all'applicazione di
analisi predittive, con lo scopo di predire, con una certa accuratezza,
il tempo rimanente prima di un successivo ``guasto'' della macchina in
esame.

\subsection{Metodi analitici di
predizione}\label{metodi-analitici-di-predizione}

Esistono fondamentalmente tre tipologie di domande alle quali la
manutenzione predittiva cerca di dare risposta. Definiamo queste domande
come Use Case, per indicare la tipologia di problema che si vuole
affrontare.

Il primo Use Case rappresenta un classico problema di classificazione.
Per problema di classificazione si intende l'identificazione della
classe di appartenenza di nuove osservazioni, sulla base di un training
set di dati che contengono istanze (osservazioni) la cui appartenenza
alle classi in esame è nota. \pagebreak

\section*{Bibliografia}\label{bibliografia}
\addcontentsline{toc}{section}{Bibliografia}

\hypertarget{refs}{}
\hypertarget{ref-maccarelliManutenzioneTutteDefinizioni}{}
Maccarelli, Marco. n.d. ``Manutenzione: tutte le definizioni delle norme
di riferimento.'' \emph{Certifico Srl}.
https://www.certifico.com/normazione/173-documenti-riservati-normazione/documenti-estratti-norme/3135-manutenzione-tutte-le-definizioni-delle-norme-di-riferimento.

\hypertarget{ref-schmidtNEXTGENERATIONCONDITION}{}
Schmidt, Bernard, Ulf Sandberg, and Lihui Wang. n.d. ``NEXT GENERATION
CONDITION BASED PREDICTIVE MAINTENANCE,'' 8.


\end{document}
